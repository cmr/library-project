%% Modified from the Use Cases Template File:

%% Created by Tom Desair (http://www.tomdesair.com)
%% Downloadable at: http://www.tomdesair.com/downloads/use-case-latex-template.zip
%% Date Modified: 03/04/2012
%
% This work may be distributed and/or modified under the
% conditions of the LaTeX Project Public License, either version 1.3
% of this license or (at your option) any later version.
% The latest version of this license is in
%   http://www.latex-project.org/lppl.txt
% and version 1.3 or later is part of all distributions of LaTeX
% version 2005/12/01 or later.

\documentclass[a4paper, 10pt, oneside, draft]{article}

\usepackage[utf8]{inputenc}
\usepackage{usecases}
\usepackage{enumerate}

\title{Library Project Requirements Analysis Document}
\date{\today}
\author{Adam Kimball, Alex Macri, Corey Richardson}

\begin{document}

\maketitle
\newpage

\section{Use Cases}

%Sometimes it is a good idea to put domain objects in \texttt{}
%The template and the descriptions are based on the book Applying UML and Patterns:
%An Introduction to Object-Oriented Analysis and Design and Iterative Development
%(3rd Edition) by Craig Larman.
\begin{usecase}

\addtitle{Use Case 1}{Handle Book Checkout \textit{(Corey Richardson)}}

%Scope: the system under design
\addfield{Scope:}{Library Management System}

%Level: "user-goal" or "subfunction"
\addfield{Level:}{User-goal}

%Primary Actor: Calls on the system to deliver its services.
\addfield{Primary Actor:}{Lab Member}

%Stakeholders and Interests: Who cares about this use case and what do they want?
\additemizedfield{Stakeholders and Interests:}{
	\item Lab Member: Wants to check out books easily, using student ID card,
        and without entering book information manually.
    \item Library Maintainer: Wants certain books to not be checked out, and
        users to only be able to checkout a certain number of books.
}

%Preconditions: What must be true on start and worth telling the reader?
%\addfield{Preconditions:}{}
%when multiple
\additemizedfield{Preconditions:}{
    \item Lab Member must have a valid student ID.
    \item Book must be in the library (cannot check out book online).
    \item Library barcode scanner is present and functional.
}

%Postconditions: What must be true on successful completion and worth telling the reader
%\addfield{Postconditions:}{}
%when multiple
\additemizedfield{Postconditions:}{
    \item Book is assigned to the user.
    \item The user's quota is decremented.
    \item The book is marked as ``unavailable''.
}

%Main Success Scenario: A typical, unconditional happy path scenario of success.
\addscenario{Main Success Scenario:}{
	\item The user browses the bookshelves.
    \item The user takes the book to the scanner.
    \item \label{itm:scancommand} The user scans the ``checkout book'' command on a command sheet.
    \item \label{itm:scanid} The user scans their student ID barcode.
    \item \label{itm:scanbarcode} The user scans the book's barcode.
    \item The scanner makes a ``success'' beep, indicating
        that the user has successfully checked out the book.
}

%Extensions: Alternate scenarios of success or failure.
\addscenario{Extensions:}{
    \item[\ref{itm:scancommand}a.] Invalid command:
        \begin{enumerate}[1.]
		\item Scanner makes a ``failure'' beep, and uses text-to-speech to
            state ``Invalid Command''.
        \item User returns to step \ref{itm:scancommand} and re-scans a
            correct command.
		\end{enumerate}

    \item[\ref{itm:scanid}a.] Unknown ID:
        \begin{enumerate}[1.]
        \item Scanner makes a ``failure'' beep, and uses text-to-speech to
            state ``Unknown ID''.
        \item User registers their ID (see ``Registration'' usecase)
        \end{enumerate}

    \item[\ref{itm:scanbarcode}a.] Unknown ISBN:
        \begin{enumerate}[1.]
        \item Scanner sends ISBN to server to register it as part of the
            library, followed by a checkout command.
        \item Server fetches the book information and adds it to the library
            database, as well as checking out that book to the user.
        \item Scanner makes a ``success'' beep.
        \end{enumerate}

    \item[\ref{itm:scanbarcode}b.] Book already checked out:
        \begin{enumerate}[1.]
        \item Scanner uses tex-to-speech to state ``Book checked out to (User
            Currently Assigned To Book), add new copy?''.
        \item User scans ``Yes'' or ``No'' command from command sheet.
        \item If Yes, server adds another copy of that book to the library and
            checks it out to the user.
        \item If No, prompt the user whether the book should be checked out.
            \begin{enumerate}[1.]
            \item If Yes, check the book out from the other user.
            \item If No, emit ``failure'' beep and use text-to-speech to state
                ``Already checked out, ask (User Currently Assigned To Book)
                to check it out''.
            \end{enumerate}
        \end{enumerate}
}

%Special Requirements: Related non-functional requirements.
\additemizedfield{Special Requirements:}{
	\item Scanner responds within 1 second 90\% of the time.
	\item Scanner resets all commands after 1 minute of waiting for input, to
        handle users walking away.
}

%Technology and Data Variations List: Varying I/O methods and data formats.
\addscenario{Technology and Data Variations List:}{
	\item All interaction done with a barcode scanner using standard user ID
        cards and ISBN barcodes.
}

%Frequency of Occurrence: Influences investigation, testing and timing of implementation.
\addfield{Frequency of Occurrence:}{Multiple times per week.}

%Miscellaneous: Such as open issues/questions
%\addfield{Open Issues:}{}

\end{usecase}


%new usecase

\newpage

\begin{usecase}

\addtitle{Use Case 2}{Handle User Book Checkin \textit{(Corey Richardson)}}
\addfield{Scope:}{Library Management System}
\addfield{Level:}{User-goal}
\addfield{Primary Actor:}{End-user}
\additemizedfield{Stakeholders and Interests:}{
    \item Lab Member: Wants to easily return a previously checked-out book.
}

\addscenario{Main Success Scenario:}{
    \item User scans ``Check In'' command.
    \item User scans user ID (this is to handle multiple copies of a book
        being signed out to multiple users).
    \item User scans book barcode.
    \item Scanner emits ``success'' beep to indicate successful completion.
}

\end{usecase}

\newpage

\begin{usecase}

\addtitle{Use Case 3}{Handle User Searching for Books \textit{(Adam Kimball)}}

%Scope: the system under design
\addfield{Scope:}{Library Management System}

%Level: "user-goal" or "subfunction"
\addfield{Level:}{User-goal}

%Primary Actor: Calls on the system to deliver its services.
\addfield{Primary Actor:}{End-user}

%Stakeholders and Interests: Who cares about this use case and what do they want?
\additemizedfield{Stakeholders and Interests:}{
	\item Lab Member: Wants to easily find a list of books presently available for checkout, with descriptions for each.
	\item Library Maintainer: Wants to have a list available without manually keeping track of each book, or having to inform Lab Members of the book's status.
}

%Preconditions: What must be true on start and worth telling the reader?
%\addfield{Preconditions:}{}
%when multiple
\additemizedfield{Preconditions:}{
	\item System is functional.
	\item System is populated with books.
}

%Postconditions: What must be true on successful completion and worth telling the reader
%\addfield{Postconditions:}{User has been served a list of books relevant to input}
%when multiple
\additemizedfield{Postconditions:}{
	\item User has been served a list of relevant books, if any exist.
	\item User has been given an empty list, if there are no books relevant to input.
}

%Main Success Scenario: A typical, unconditional happy path scenario of success.
\addscenario{Main Success Scenario:}{
	\item \label{itm:webpage} User visits the Library webpage.
	\item \label{itm:search} User types a book title or author into text field, and clicks 'search'.
	\item \label{itm:listing} User is returned a paginated list of all books containing that text in their title or author name, with a truncated version of their description for each, and whether or not each is available for loan.
}

%Extensions: Alternate scenarios of success or failure.
\addscenario{Extensions:}{
	\item[\ref{itm:search}a.] User selects 'All Books' link, rather than inputting a search term:
		\begin{enumerate}[1.]
		\item User selects 'All Books'.
		\item User is presented with a paginated listing of all books from the book database, a truncated version of their description, and whether or not they are available for loan.
		\end{enumerate}
	\item[\ref{itm:listing}a.]User selects a book:
		\begin{enumerate}[1.]
		\item User clicks on a book in the list.
		\item User is redirected to a page with large cover image(if available), and a full description.
		\end{enumerate}
	\item[\ref{itm:listing}b.] User does not input a search that has a result:
		\begin{enumerate}[1.]
		\item No books or authors match user's search.
		\item User is returned an empty list.
		\end{enumerate}
}

%Special Requirements: Related non-functional requirements.
\additemizedfield{Special Requirements:}{
	\item Search results returned by server within 5 seconds 90\% of the time.
	\item Books with no available cover image have an implemented placeholder image.
	\item List served is clean/easy to read.
}

%Technology and Data Variations List: Varying I/O methods and data formats.
\additemizedfield{Technology and Data Variations List:}{
	\item Accessible with recent versions of IE, Chrome, Firefox, and text-based browsers such as Lynx/Elinks.
}
%Frequency of Occurrence: Influences investigation, testing and timing of implementation.
\addfield{Frequency of Occurrence:}{Multiple times per week.}

%Miscellaneous: Such as open issues/questions
%\addfield{Open Issues:}{}

\end{usecase}

\newpage


\begin{usecase}

\addtitle{Use Case 4}{Viewing Meeting Videos \textit{(Adam Kimball)}}
\addfield{Scope:}{Clarkson Open Source Institute Wiki}
\addfield{Level:}{User-goal}
\addfield{Primary Actor:}{End-user}
\additemizedfield{Stakeholders and Interests:}{
    \item Lab Member: Wants to easily view recorded lab meetings or meeting minutes they may have missed.
}

\addscenario{Main Success Scenario:}{
    \item User accesses video portion of the COSI Wiki site.
    \item User selects a meeting date
    \item User is presented with an embedded video player, as well as a small
        summary of what happened at the meeting
}

\end{usecase}


\newpage


\begin{usecase}

\addtitle{Use Case 5}{Interfacing with Database \textit{(Alex Macri)}}

%Scope: the system under design
\addfield{Scope:}{Library Management System}

%Level: "user-goal" or "subfunction"
\addfield{Level:}{User-goal}

%Primary Actor: Calls on the system to deliver its services.
\addfield{Primary Actor:}{Scanner and Web Interface}

%Stakeholders and Interests: Who cares about this use case and what do they want?
\additemizedfield{Stakeholders and Interests:}{
	\item  User:
	Wants to easily find a book with the description and if the book is checkout and if the book is checkout when will the be returned.
	See valuable information in a timely fashion.
	Usable functionality of the database so it is easy to use, understand, and works as expected.
	\item Maintainer:
	For modularity so it is easier to add more functions when needed in the future.
	The database to be fast with searching through the list of books.
}

%Preconditions: What must be true on start and worth telling the reader?
%\addfield{Preconditions:}{}
%when multiple
\additemizedfield{Preconditions:}{
	\item System is functional.
	\item System is populated with books.
	\item System has functionality for user requests from the browser.
}

%Postconditions: What must be true on successful completion and worth telling the reader
%\addfield{Postconditions:}{\textit{None}}
%when multiple
\additemizedfield{Postconditions:}{
    \item The Database preforms all tasks correctly
    \item When requested to list the books. The System will return the either error or on success return the list in a proper format.
    \item When requested to add a book. The System with add the book with the correct information for the book.
    \item When requested to delete a book. The System with delete the book with the correct information for the book.
    \item When requested to checkout and checkin a book. The System with a flag on whether the book is checked-in or checked-out the book with the correct information for the book.

}

%Main Success Scenario: A typical, unconditional happy path scenario of success.
\addscenario{Main Success Scenario:}{
	\item User visits the Library webpage.
	\item User selects a function that with interact with the database.
	\item User is presented functionality of these procedures and can complete the task required.
	\item User completes functionality of the database in 1 sec.
}

%Extensions: Alternate scenarios of success or failure.
\addscenario{Extensions:}{
	\item[1a.] Invalid login data:
        \begin{enumerate}[1.]
		\item System shows failure message.
		\item User cannot login and prompts the user to retry.
		\end{enumerate}
	\item[1b.] Valid login data:
        \begin{enumerate}[1.]
		\item System shows success message.
		\item User continues onto the site.
		\end{enumerate}
	\item[2a.] Invalid checkout data:
        \begin{enumerate}[1.]
		\item[1.] System shows failure message.
		\end{enumerate}
	\item[2b.] Valid checkout data:
        \begin{enumerate}[1.]
		\item System shows success message.
		\item User has checked out out the book and the database is modified.
		\end{enumerate}
	\item[3a.] Invalid checkin data:
        \begin{enumerate}[1.]
		\item System shows failure message.
		\end{enumerate}
	\item[3b.] Valid checkin data:
        \begin{enumerate}[1.]
		\item System shows success message.
		\item User has checked in out the book and the database is modified.
		\end{enumerate}
	\item[4a.] Invalid delete data:
        \begin{enumerate}[1.]
		\item System shows failure message.
		\end{enumerate}
	\item[4b.] Valid delete data:
        \begin{enumerate}[1.]
		\item System shows success message
		\item User updates the database with the book entry gone.
		\end{enumerate}
	\item[5a.] Invalid add data:
        \begin{enumerate}[1.]
		\item System shows failure message.
		\end{enumerate}
	\item[5b.] Valid add data:
        \begin{enumerate}[1.]
		\item System shows failure message.
		\item User adds a book to the database.
		\end{enumerate}
	\item[6a.] Invalid list data:
        \begin{enumerate}[1.]
		\item System shows failure message.
		\end{enumerate}
	\item[6b.] Valid list data:
        \begin{enumerate}[1.]
		\item System shows success message.
		\item User list the books that are currently in the database.
		\end{enumerate}
}

%Special Requirements: Related non-functional requirements.
\additemizedfield{Special Requirements:}{
	\item Response to request in a fast and reasonable timeframe.
}

%Technology and Data Variations List: Varying I/O methods and data formats.
\addscenario{Technology and Data Variations List:}{
	\item Implemented with the Rust programming language.
	\item Uses PostgreSQL database server.
	\item Runs on standard Linux systems.
}

%Frequency of Occurrence: Influences investigation, testing and timing of implementation.
\addfield{Frequency of Occurrence:}{
	Multiple time per week.
}

%Miscellaneous: Such as open issues/questions
%\addfield{Open Issues:}{}

\end{usecase}


\begin{usecase}

\addtitle{Use Case 6}{Loading from Database \textit{(Alex Macri)}}
\addfield{Scope:}{Library Management System}
\addfield{Level:}{User-goal}
\addfield{Primary Actor:}{User of System}
\additemizedfield{Stakeholders and Interests:}{
    \item Lab Member: Wants to easily be able to query checked-in or checked-out books.
    \item Lab Member: Wants to easily change the check in and check out status.
}

\addscenario{Main Success Scenario:}{
    \item System gets the request and changes the flag for the book specified or returns error if invalid book
    \item System return success message if the procedure in successful
}

\end{usecase}



\end{document}
