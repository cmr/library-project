%% Modified from the Use Cases Template File:

%% Created by Tom Desair (http://www.tomdesair.com)
%% Downloadable at: http://www.tomdesair.com/downloads/use-case-latex-template.zip
%% Date Modified: 03/04/2012
%
% This work may be distributed and/or modified under the
% conditions of the LaTeX Project Public License, either version 1.3
% of this license or (at your option) any later version.
% The latest version of this license is in
%   http://www.latex-project.org/lppl.txt
% and version 1.3 or later is part of all distributions of LaTeX
% version 2005/12/01 or later.

\documentclass[a4paper, 10pt, oneside, draft]{article}

\usepackage[utf8]{inputenc}
\usepackage{usecases}
\usepackage{enumerate}

\title{Library Project Requirements Analysis Document}
\date{\today}
\author{Adam Kimball, Alex Macri, Corey Richardson}

\begin{document}

\maketitle
\newpage

\section{Use Cases}

%Sometimes it is a good idea to put domain objects in \texttt{}
%The template and the descriptions are based on the book Applying UML and Patterns:
%An Introduction to Object-Oriented Analysis and Design and Iterative Development
%(3rd Edition) by Craig Larman.
\begin{usecase}

\addtitle{Use Case 1}{Handle Book Checkout \textit{(Corey Richardson)}}

%Scope: the system under design
\addfield{Scope:}{Library Management System}

%Level: "user-goal" or "subfunction"
\addfield{Level:}{User-goal}

%Primary Actor: Calls on the system to deliver its services.
\addfield{Primary Actor:}{Lab Member}

%Stakeholders and Interests: Who cares about this use case and what do they want?
\additemizedfield{Stakeholders and Interests:}{
	\item Lab Member: Wants to check out books easily, using student ID card,
        and without entering book information manually.
    \item Library Manager: Wants certain books to not be checked out, and
        users to only be able to checkout a certain number of books.
}

%Preconditions: What must be true on start and worth telling the reader?
%\addfield{Preconditions:}{}
%when multiple
\additemizedfield{Preconditions:}{
    \item Lab Member must have a valid student ID.
    \item Book must be in the library (cannot check out book online).
}

%Postconditions: What must be true on successful completion and worth telling the reader
%\addfield{Postconditions:}{}
%when multiple
\additemizedfield{Postconditions:}{
    \item Book is assigned to the user.
    \item The user's quota is decremented.
    \item The book is marked as ``unavailable''.
}

%Main Success Scenario: A typical, unconditional happy path scenario of success.
\addscenario{Main Success Scenario:}{
	\item The user browses the bookshelves.
    \item The user takes the book to the scanner.
    \item \label{itm:scancommand} The user scans the ``checkout book'' command on a command sheet.
    \item \label{itm:scanid} The user scans their student ID barcode.
    \item \label{itm:scanbarcode} The user scans the book's barcode.
    \item \label{itm:happybeep} The scanner makes a ``success'' beep, indicating
        that the user has successfully checked out the book.
}

%Extensions: Alternate scenarios of success or failure.
\addscenario{Extensions:}{
    \item[\ref{itm:scancommand}.a] Invalid command:
		\begin{enumerate}
		\item Scanner makes a ``failure'' beep, and uses text-to-speech to
            state ``Invalid Command''.
		\item User returns to step 2 and corrects the errors
		\end{enumerate}
}

%Special Requirements: Related non-functional requirements.
\additemizedfield{Special Requirements:}{
	\item first applicable non-functional requirement
	\item second applicable non-functional requirement
}

%Technology and Data Variations List: Varying I/O methods and data formats.
\addscenario{Technology and Data Variations List:}{
	\item[1a.] Alternative first action with other technology
}

%Frequency of Occurrence: Influences investigation, testing and timing of implementation.
\addfield{Frequency of Occurrence:}{}

%Miscellaneous: Such as open issues/questions
%\addfield{Open Issues:}{}

\end{usecase}


%new usecase

\newpage

\begin{usecase}

\addtitle{Use Case 2}{Handle User Searching for Books \textit{(Adam Kimball)}}

%Scope: the system under design
\addfield{Scope:}{Library Management System}

%Level: "user-goal" or "subfunction"
\addfield{Level:}{User-Goal}

%Primary Actor: Calls on the system to deliver its services.
\addfield{Primary Actor:}{End-User}

%Stakeholders and Interests: Who cares about this use case and what do they want?
\additemizedfield{Stakeholders and Interests:}{
	\item Lab Member: Wants to easily find a list of books presently available for checkout, with descriptions for each.
	\item Library Maintainer: Wants to have a list available without manually keeping track of each book, or having to inform Lab Members of the book's status.
}

%Preconditions: What must be true on start and worth telling the reader?
%\addfield{Preconditions:}{}
%when multiple
\additemizedfield{Preconditions:}{
	\item Webserver is functional.
	\item Database is populated with books.
}

%Postconditions: What must be true on successful completion and worth telling the reader
%\addfield{Postconditions:}{User has been served a list of books relevant to input}
%when multiple
\additemizedfield{Postconditions:}{
	\item User has been served a list of relevant books, if any exist.
	\item User has been given an empty list, if there are no books relevant to input.
}

%Main Success Scenario: A typical, unconditional happy path scenario of success.
\addscenario{Main Success Scenario:}{
	\item \label{itm:webpage} User visits the Library webpage.
	\item \label{itm:search} User types a book title or author into text field, and clicks 'search'.
	\item \label{itm:listing} User is returned a paginated list of all books containing that text in their title or author name, with a truncated version of their description for each, and whether or not each is available for loan.
}

%Extensions: Alternate scenarios of success or failure.
\addscenario{Extensions:}{
	\item[\ref{itm:search}.a] User selects 'All Books' link, rather than inputting a search term:
		\begin{enumerate}[1]
		\item User selects 'All Books'.
		\item User is presented with a paginated listing of all books from the book database, a truncated version of their description, and whether or not they are available for loan.
		\end{enumerate}
	\item[\ref{itm:listing}.a]User selects a book:
		\begin{enumerate}[1]
		\item User clicks on a book in the list.
		\item User is redirected to a page with large cover image(if available), and a full description.
		\end{enumerate}
	\item[\ref{itm:listing}.b] User does not input a search that has a result:
		\begin{enumerate}[1]
		\item No books or authors match user's search.
		\item User is returned an empty list.
		\end{enumerate}
}

%Special Requirements: Related non-functional requirements.
\additemizedfield{Special Requirements:}{
	\item Search results returned by server within 5 seconds 90\% of the time.
	\item Books with no available cover image have an implemented placeholder image.
	\item Interface is simple, and easy to navigate.
	\item List served is clean/easy to read.
}

%Technology and Data Variations List: Varying I/O methods and data formats.
\additemizedfield{Technology and Data Variations List:}{
	\item Accessible with recent versions of IE, Chrome, Firefox, and text-based browsers such as Lynx/Elinks.
}
%Frequency of Occurrence: Influences investigation, testing and timing of implementation.
\addfield{Frequency of Occurrence:}{Multiple times per week.}

%Miscellaneous: Such as open issues/questions
%\addfield{Open Issues:}{}

\end{usecase}



\end{document}
